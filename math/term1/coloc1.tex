\documentclass{article}
\usepackage[utf8]{inputenc}
\usepackage[T2A]{fontenc}
\usepackage[russian]{babel}
\usepackage{amsfonts}
\usepackage{amsmath}
\usepackage{amssymb}
\usepackage{arcs}
\usepackage{fancyhdr}
\usepackage{float}
\usepackage[left=3cm,right=3cm,top=3cm,bottom=3cm]{geometry}
\usepackage{graphicx}
\usepackage{hyperref}
\usepackage{multicol}
\usepackage{stackrel}
\usepackage{xcolor}

\begin{document}
\pagestyle{empty}
\normalsize

\section{Билеты}
\subsection{Законы де Моргана}
Пусть $\{X_{\alpha}\}_{\alpha \in A}$ - семейство множеств, $Y$ - ммножесттво. Тогда
\begin{equation}
	Y \setminus \displaystyle\bigcup\limits_{\alpha \in A}X_{\alpha} = \displaystyle \bigcap\limits_{\alpha \in A}(Y \setminus X_{\alpha})
\end{equation}
\begin{equation}
	Y \setminus \displaystyle \bigcap\limits_{\alpha \in A}X_{\alpha} = \displaystyle \bigcup\limits_{\alpha \in A}(Y \setminus X_{\alpha})
\end{equation}
\begin{equation}
	Y \cap \displaystyle\bigcup\limits_{\alpha \in A}X_{\alpha} = \displaystyle\bigcup\limits_{\alpha \in A}(Y \cap X_{\alpha})
\end{equation}
\begin{equation}
	Y \cup \displaystyle\bigcap\limits_{\alpha \in A}X_{\alpha} = \displaystyle\bigcap\limits_{\alpha \in A}(Y \cup X_{\alpha})
\end{equation}

\subsection{Аксиомы порядка и их элемеенетарные следствия. Два способа расширения вещественной прямой.}
Между элементами $\mathbb{R}$ определено отношение $\leq$ со следующими свойствами:
\begin{enumerate}
	\item $\forall x, y: x \leq y \vee y \leq x$
	\item $x \leq y \wedge y \leq z \Rightarrow x \leq z$
	\item $x \leq y \wedge y \leq x \Rightarrow x = y$
	\item $x \leq y \Rightarrow x + z \leq y + z \forall z$
	\item $0 \leq x \wedge 0 \leq y \Rightarrow 0 \leq xy$
\end{enumerate}

\noindent$\mathbb{R}$ можно расширять с помощью $+\infty$ и $-\infty$.  

\subsection{Модуль числа}
{\bfОпределение.} Пусть $x \in \mathbb{R}$. Число
\begin{equation*}
	|x| = \begin{cases}
		x, & x \geq 0, \\
		-x, & x < 0
	\end{cases}
\end{equation*}
называется {\bf модулем} или {\bf абсолютной величиной} числа $x$. 

\noindent Свойства модуля:
\begin{enumerate}
	\item $|-x| = x$
	\item $\pm x \leq |x|$
	\item $|xy| = |x||y|$
	\item $\displaystyle|\frac{x}{y}| = \displaystyle\frac{|x|}{|y|}, y \neq 0$
	\item $\displaystyle||x| - |y|| \leq |x \pm y| \leq |x| + |y|$
	\item $|x| < a \Leftrightarrow -a < x < a,\ a > 0$
\end{enumerate}

\subsection{Комплексные числа}

{\bf Определение.} Комплексное число - это упорядоченая пара вещественных чисел $(x, y)$, так что множество $\mathbb{C} = \mathbb{R}^2$.

Определим некоторые свойства для комплексных чисел. Пусть $z_1 = (x_1, y_1),\ z_2 = (x_2, y_2)$, тогда:
\begin{enumerate}
	\item $z_1 + z_2 = (x_1 + x_2, y_1 + y_2)$
	\item $0 = (0, 0)$
	\item $-z = (-x, -y)$
	\item $z_1 \cdot z_2 = (x_1x_2 - y_1y_2, x_1y_2 + x_2y_1)$
	\item $1 = (1, 0)$
	\item $\mathbb{R} = (x, 0), \mathbb{R} \subset \mathbb{C}$
	\item $i = (0, 1)$ - мнимая еденица
	\item $(0, y)$ - мнимые числа
	\item $z = (x, y) = x(1, 0) + y(0, 1) = x + iy$
	\item $\displaystyle\frac{1}{z} = \displaystyle\frac{x}{x^2 + y^2} + i\displaystyle\frac{-y}{x^2 + y^2}$
	\item $x = Rez, y = Imz$
\end{enumerate}

\noindent $\overline{z} = x - iy$ - сопряжённое к $z$. Тогда
\begin{equation*}
	Rez = \frac{z + \overline{z}}{2}, \qquad Imz = \frac{z - \overline{z}}{2i}
\end{equation*}
\begin{equation*}
	\overline{\overline{z}} = z, \qquad \overline{z_1 \cdot z_2} = \overline{z_1} \cdot \overline{z_2}
\end{equation*}
Определим {\bf модуль} комплексного числа:
\begin{equation*}
	|z| = \sqrt{x^2 + y^2}
\end{equation*}
Он обладает следующимии свойствами:
\begin{enumerate}
	\item $|z_1z_2| = |z_1||z_2|$
	\item $\displaystyle\Big|\displaystyle\frac{z_1}{z_2}\Big| = \displaystyle\frac{|z_1|}{|z_2|}$
	\item $\big| |z_1| - |z_2| \big| \leq | z_1 \pm z_2| \leq |z_1| + |z_2|$
	\item $z\overline{z} = |z|^2$
\end{enumerate}


\noindent {\bf Определение.} Тригонометрическая форма комплекесного числа определяется так - $z = r(\cos\phi + i\sin\phi),\quad r = |z|,\ \phi=\arg z$.


\noindent {\bf Формула Муавра.}
\begin{equation*}
	z^n = r^n(\cos n\phi + i\sin n\phi)
\end{equation*}


\noindent {\bf Определение.} Показзательной формой комлексного число наззывается $z = r \cdot e^{i\phi}, \quad e^{i\phi} = \cos\phi = i\sin\phi$

\noindent {\bf Определение.} Расширенная комплексная плоскость - $\hat{\mathbb{C}} = \mathbb{C} \cup \{\infty\}$


\subsection{Принцип математической индукции. Индуктивные множества. Неравенство Бернулли.}

\noindent {\bf Определение.} Пусть $\{\mathbb{P}\}$ - последовательность утверждений. Если
\begin{enumerate}
	\item $\mathbb{P}_1$ верно (база индукции)
	\item $\forall a \in \mathbb{N} \mathbb{P}_n \Rightarrow \mathbb{P}_{n + 1}$ (индукционный переход, $\mathbb{P}_n$ - индукционное предположение)
\end{enumerate}
Тогда $\mathbb{P}_n$ верно $\forall n \in \mathbb{N}$.


\noindent {\bf Определение.} Множество $M \in \mathbb{R} $ называется {\bf индуктивным}, если $1 \in M$ и $\forall x \in M \ x + 1 \in M$.


\noindent {\bf Теорема (Неравенство Бернулли). }
\begin{equation*}
	(1 + x)^n \geq 1 + nx \forall n \geq 1 \in \mathbb{N}
\end{equation*} 

\subsection{Бином Ньютона.}
{\bf Теорема.} Если $n \in \mathbb{Z}_+,\ x,y \in \mathbb{R} \ \text{или} \ \mathbb{C}, \text{то}$ 
\begin{equation*}
	(x + y)^n = \displaystyle\sum\limits_{k = 0}^{n}C_n^kx^ky^{n-k}
\end{equation*}

\subsection{Аксиома Архимеда. Плотность множества рациональных чисел в $\mathbb{R}$.}
{\bf Аксиома.} $\forall x,y > 0 \in \mathbb{R},\ \exists n \in \mathbb{N} : nx > y$

\noindent{\bf Плотность. } $\forall a<b \in \mathbb{R} \ \exists c: a<c<b$

\subsection{Ограниченные, односторонне ограниченные множества. Теорема о максимуме и минимуме конечного множества и следствия из нее.}
{\bf Определение.} Пусть $X \subset \mathbb{R}$. Если $\exists b \in \mathbb{R}\ \ \forall x \in X:\ x \leq b$, множество $X$ называется огрнаичнным сверху. 

\noindent {\bf Определение.} Пусть $X \subset \mathbb{R}$. Если $\exists b \in \mathbb{R}\ \ \forall x \in X:\ x \geq b$, множество $X$ называется огрнаичнным снизу. 

\noindent {\bf Определение.} Пусть $X \subset \mathbb{R}$. Если $X$ ограниччено сверху и снизу, то оно называется огрниченным.

\noindent {\bf Теорема. } $\forall$ конечного множества $X \subset \mathbb{R}\ \exists \max(X), \min(X)$

\noindent {\bf Следствия.}
\begin{enumerate}
	\item $X$ ограничена сверху $\Rightarrow \exists\ max(X)$
	\item $X$ ограничена снизу $\Rightarrow \exists\ min(X)$
\end{enumerate}

\subsection{Инъекции, сюръекции, биекции, обратные отображения. Эквивалентные множества, свойства эквивалентности множеств. Примеры эквивалентных множеств.}
\noindent {\bf Определение.} $f: A \to B$ называется инъекцией, если верно $f(x_1) = f(x_2) \Rightarrow x_1 = x_2$

\noindent {\bf Определение.} $f: A \to B$ называется сюръекцией, если верно $\forall y \in B\  \exists x \in A : f(x) = y$

\noindent {\bf Определение.} $f: A \to B$ называетсяс биекцией, еслли оно инъективно и сюръективно, то есть
\begin{equation*}
	\begin{cases}
		f(x_1) = f(x_2) \Rightarrow x_1 = x_2 \ \forall x_1,x_2 \in A\\
		\forall y \in B \ \exists x \in A : f(x) = y
	\end{cases}
\end{equation*} 


\noindent {\bf Определение.} $g: B \to A$ называется обратным к $f: A \to B$, если
\begin{equation*}
	\begin{cases}
		g \circ f = Id_A \\
		f \circ G = Id_B
	\end{cases}
\end{equation*}

\noindent {\bf Определение.} Два множеста $A$ и $B$ эквивалентны, если можно построить между их элементами отображение $f: A \to B$ такое, что оно будет биективным.

\noindent Пусть $\sim$ - отношение эквивалентности. Тогда его {\bf свойства} выглядят так:
\begin{enumerate}
	\item $A \sim S$ (рефлексивность).
	\item $A \sim B \Rightarrow B \sim A$ (симметричность).
	\item $A \sim B \wedge B \sim C \Rightarrow A \sim C$ (транзитивность).
\end{enumerate}

\noindent {\bf Примеры.}
\begin{enumerate}
	\item $\mathbb{N} \sim \mathbb{Z}$
	\item $[a, b] \sim [a + h,b + h] \ \forall a, b, h \in \mathbb{R}$
	\item $(0, 1) \sim (1, +\infty)$, так как $[x \in (0, 1)] \leftrightarrow [ y = \displaystyle\frac{1}{x} \in (1, +\infty)]$
\end{enumerate}

\subsection{Счетные множества, не более чем счетные множества (определения и примеры). Образ счетного множества.}
\noindent {\bf Определение.} множество $X$ называется счёным, если $\exists f: N \to X$, которое является биекцией.

\noindent {\bf Примеры.} $\mathbb{N}, \mathbb{Z}, \mathbb{Q}, \mathbb{A}$

\noindent {\bf Определение.} Непустое множество являющееся конечным или счётным называется не более, чем счётным. 

\noindent {\bf Теорема.} При любом отображении образ счётного множества конечен или счётен.

\subsection{Две теоремы о счетных подмножествах.}

\noindent {\bf Теорема.} Любое подмножество счетного множества не более чем счетно.

\noindent {\bf Теорема.} Любое бесконечное множество содержит счетное подмножество.


\subsection{Утверждения о произведении счетных множеств и о не более чем счетном объединении не более чем счетных множеств. Счетность множества рациональных чисел}

\noindent {\bf Теорема.} Не более чем счётное (конечное или счётное) объединение не более чем счётных множеств является не более чем счётным множеством.

\noindent {\bf Теорема.} Декартово произведение двух счётных множеств $A \times B$ cчётно.

\noindent {\bf Теорема.} Множество всех действительных чисел несчетно. 

\subsection{Аксиома Кантора. Несчетность отрезка и некоторых других множеств вещественных чисел.}

\noindent {\bf Аксиома Кантора.} Любая последовательность вложенных друг в друга отрезков, длины которых стремятся к нулю, имеет одну общую точку.

\noindent {\bf Теорема.}  Любые два конечных интервала (соответственно отрезка) числовой прямой равномощны. Если заданы два интервала $(a,b)$ и $(c,d)$, то отображение.
\begin{equation*}
		x = \frac{(d - c)t = bc - ad}{b - a}, a < t < b
\end{equation*}
является биекцией интервалов $(a,b)$ и $(c,d)$ (соответственно отрезков $[a,b]$ и $[c,d]$).

\subsection{Единственность предела последовательности и ограниченность сходящейся последовательности.}
\noindent {\bf Теорема.} Последовательность точек расширенной числовой прямой $\overline{\mathbb{R}}$ может иметь на этой прямой только один предел.

\noindent {\bf Теорема.} Если числовая последовательность имеет конечный предел, то она ограничена.

\subsection{Утверждение о предельном переходе в неравенствах и следствия из нее. Теорема о сжатой последовательности.}
\noindent {\bf Утверждение.} $x_n = a \in \overline{\mathbb{R}} \Leftrightarrow \lim\limits_{n \to \infty}x_n = a\  \forall n \in \mathbb{N}$

\noindent {\bf Теорема (о двух милиционерах).}
\begin{equation*}
	\begin{cases}
		x_n \leq y_n \leq z_n,\ \forall n \in \mathbb{N} \\
		\lim\limits_{n \to \infty}x_n = \lim\limits_{n \to \infty}z_n = a \in \overline{\mathbb{R}}
	\end{cases}
	\Leftrightarrow \lim\limits_{n \to \infty}y_n = a,\ \forall x_n, y_n, z_n = \overline{\mathbb{R}}
\end{equation*}

\subsection{Бесконечно малые последовательности, элементарные свойства класса бесконечно-малых последовательностей, в т.ч. лемма о произведении б.м. на ограниченную. Бесконечно большие последовательности. Связь между б.б. и б.м. последовательностями.}
\noindent {\bf Определение.} $\lim\limits_{n \to \infty}a_n = 0$ - бесконечно малая последовательность.

\noindent {\bf Свойства.}
\begin{enumerate}
	\item Сумма двух бесконечно малых последовательностей есть бесконечно малая последовательность.
	\item Разность двух бесконечно малых последовательностей есть бесконечно малая последовательность.
	\item Бесконечно малая последовательность ограничена.
	\item Произведение ограниченной последовательности на бесконечно малую есть бесконечно малая последовательность.
	\item Если все элементы бесконечно малой последовательности, начиная с некоторого номера, равны одному и тому же числу, то это число - ноль.
	\item Если $\{x_{n}\}$ - бесконечно большая последовательность, то начиная с некоторого номера определена последовательность $\{\frac{1}{x_{n}}\}$ , причём она является бесконечно малой.
	\item Если $\{y_{n}\}$ - бесконечно малая последовательность, то начиная с некоторого номера определена последовательность $\{\frac{1}{y_{n}}\}$ , причём она является бесконечно большой.
\end{enumerate}

\noindent {\bf Определение.} Бесконечно большая — числовая последовательность, стремящаяся к (предел которой равен) бесконечности определённого знака.

\subsection{Теорема об арифметических действиях над сходящимимся последовательностями.}

\noindent{Теорема.} Пусть $\lim\limits_{n \to \infty}x_n = a$ и $\lim\limits_{n \to \infty}y_n = b$, то:
\begin{enumerate}
	\item $\lim\limits_{n \to \infty}(x_n + y_n) = a + b$
	\item $\lim\limits_{n \to \infty}(x_ny_n) = ab$
	\item $\lim\limits_{n \to \infty}\frac{x_n}{y_n} = \frac{a}{b},\ b \neq 0$
\end{enumerate}

\subsection{Теорема об арифметических действиях над бесконечно большими последовательностями.}
\noindent {\bf Теорема.}
\begin{enumerate}
	\item $x_n \to +\infty$, $y_n$ ограничена снизу $\Rightarrow x_n + y_n \to +\infty$
	\item $x_n \to -\infty$, $y_n$ ограничена сверху $\Rightarrow x_n + y_n \to -\infty$
	\item $x_n \to \infty$, $y_n$ ограничена $\Rightarrow x_n + y_n \to \infty$
	\item $x_n \to +\infty, \exists\delta > 0:\ \forall n y_n > \delta \Rightarrow x_ny_n \to +\infty$
	\item $x_n \to -\infty, \exists\delta > 0:\ \forall n y_n > \delta \Rightarrow x_ny_n \to -\infty$
	\item $x_n \to \infty, \exists\delta > 0:\ \forall n y_n > \delta \Rightarrow x_ny_n \to \infty$
	\item $x_n \to x \in \mathbb{R} y_n \to \infty \Rightarrow \frac{x_n}{y_n} \to 0$
	\item $x_n \to x \in \hat{\mathbb{R}}\setminus\{0\}, y_n \to 0 \Rightarrow \frac{x_n}{y_n} \to \infty$
	\item $x_n \to \infty, y_n \to y \in \mathbb{R} \Rightarrow  \frac{x_n}{y_n} \to \infty$
\end{enumerate}

\subsection{Теорема о стягивающихся отрезках}
\noindent {\bf Теорема.} Если $[x_1, y_1], [x_2, y_2], \cdots, [x_n, y_n]$ - последовательность стягивающихся отрезков, то $\exists!$ точка, принадлежащия этим отрезков.
\begin{equation*}
	\displaystyle\bigcap\limits_{n = 1}^{\infty}[x_n, y_n] = c, c \in \mathbb{R}
\end{equation*}

\subsection{Теорема о существовании супремума и инфимума. Элементарные свойства супремума и инфимума}

\noindent {\bf Определение.} Верхняя грань множества называется $\sup\limits_{x \in X}x$ (supremum)

\noindent {\bf Определение.} Нижняя грань множества называется $\inf\limits_{x \in X}x$ (infinum)

\noindent {\bf Теорема. } Всякое ограниченное сверху непустое числовое множество имеет верхнюю грань, а всякое ограниченное снизу непустое числовое множество имеет нижнюю грань.

\noindent{\bf Замечания}
\begin{enumerate}
	\item Если множество неограничено сверху, то $\sup X = +\infty$
	\item Если множество неограничено снизу, то $\inf X = -\infty$
\end{enumerate}

\noindent{\bf Свойства.}
\begin{enumerate}
	\item $\sup(A + B) = \sup A + \sup B$
	\item $\sup(tA) = t\sup A\ \forall t > 0$
	\item $\sup(-A) = -\inf(A)$
\end{enumerate}

\subsection{Теорема о пределе монотонной последовательности.}
\noindent {\bf Теорема (Вейерштрасса).}
\begin{enumerate}
	\item Любая возрастающая (убывающая) последовательность в $\mathbb{R}$ имеет предел в $\overline{\mathbb{R}}$, и он равен её супремуму (инфимуму).
	\item Любая возрастающая (убывающая) и ограниченная сверху (снизу) последовательность сходится.
	\item Любая ограниченная монотонная последовательность сходится.
\end{enumerate}

\subsection{\bf Определение числа $e$, соответствующий замечательный предел}
\noindent {\bf Определение.} 
\begin{equation*}
	\lim\limits_{n \to \infty}(1 + \frac{1}{n})^n = \lim\limits_{n \to \infty}(1 + \frac{1}{n})^{n + 1} = e
\end{equation*}

\subsection{Формула Герона.}
\noindent{\bf Формула.} Пусть $a > 0, x_0 > 0$
\begin{equation*}
	x_{n+1} = \frac{1}{2}(x_n + \frac{a}{x_n}), n \in \mathbb{Z}_+
\end{equation*}
Тогда $x_n \to \sqrt{a}$

\subsection{Две леммы о пределе подпоследовательности}
\noindent{\bf Лемма.} Пусть $\{x_n\}_n$ - числовая последовательность $x_n \to x \in \overline{\mathbb{R}}$.
Тогда $forall$ последовательности $\{x_{n_k}\}_k\ \exists \lim\limits_{n \to  \infty}x_{n_k} = x$

\noindent{\bf Лемма.} Пусть $\{n_k\}_k, \{m_j\}_j$ - строго возрастающая последовательность в $\mathbb{N}$
Тогда, если $\{n_k\}_k \cup \{m_j\}_j = \mathbb{N}$,
\begin{equation*}
	\{x_n\} \text{ - числовая последовательность }\quad x_{n_k} \to x \quad x_{mj} \to x \text{, то } x_n \to x
\end{equation*}
\end{document}
