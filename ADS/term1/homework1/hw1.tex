\documentclass{article}
\usepackage[utf8]{inputenc}
\usepackage[russian]{babel}
\usepackage{amsmath}
\usepackage{amssymb}
\usepackage{graphicx}

	

\title{Домашняя работа №1}
\author{Михаил Корнилович}
\date{Сентябрь 2022}

\begin{document}
\pagenumbering{gobble}
\maketitle
\newpage
\pagenumbering{arabic}
\section{Задача №1}

\begin{equation*}
    \begin{cases}
        f_1(n) \leq c_1g_1(n) \\
        f_2(n) \leq c_2g_2(n) \\
    \end{cases}
    \Rightarrow
    f_1(n) + f_2(n) \leq c_1g_1(n) + c_2g_2(n)
    \Rightarrow
    f_1(n) + f_2(n) \leq \mathcal{O}(g_1(n) + g_2(n))
\end{equation*}

\section{Задача №2}

Имеем по определению:

$f(n) \leq c_1f(n)$ и $g(n) \leq c_2g(n)$

\noindent Также:

$min(f(n), g(n)) \leq max(f(n), g(n)) \leq max(f(n), g(n))$

\noindent Распишем это неравенство, используя определение:

$min(c_1f(n), c_2g(n)) \leq max(f(n), g(n)) \leq max(c_1f(n), c_2g(n))$

\noindent Полчуается  $\exists c_1,c_2: max(f(n), g(n))$ ограничена сверху и снизу $\Rightarrow$

$\Rightarrow max(f(n), g(n)) = \Theta(f(n) + g(n))$

\section{Задача №3}
\noindent Чтобы это равенство оказалось верным, нужно доказать, что:

$\sum_{i=1}^{n+5} 2^i \leq c\cdot2^n$
 
\noindent Воспользуемся математической индукцией.

\noindent База индукции:

Пусть $n_0 = 0$, а $c = 62$, тогда $62 \leq 62$ - верно

\noindent Индукционный переход:

Пусть для n неравенство $\sum_{i=1}^{n+5} 2^i \leq c\cdot2^n$ верно.

Проверим его для n + 1:
\begin{equation*}
    \sum_{i=1}^{n+6} 2^i \leq c\cdot2^{(n+1)}
    \Leftrightarrow
    \sum_{i=1}^{n+5} 2^i + 2^{n+6} \leq 2c\cdot2^n
    \Leftrightarrow
    \sum_{i=1}^{n+5} 2^i + 2^{n+6} \leq c\cdot2^n + c\cdot2^n
\end{equation*}
\noindentТ.к. $\sum_{i=1}^{n+5} 2^i \leq c\cdot2^n$ верно, нам надо проверить только то, что $2^{n+6} \leq c\cdot2^n$

\begin{equation*}
    2^{n+6} \leq c\cdot2^n 
    \Leftrightarrow
    2^6\cdot2^n \leq c\cdot2^n
    \Leftrightarrow
    2^6 \leq c
\end{equation*}

\noindent Последнее неравенство верно, т.к. c - любое число, следовательно

$\sum_{i=1}^{n+5} 2^i = \mathcal{O}(2^n)$
\section{Задача №4}

Чтобы равенство оказалочь верным нужно доказать, что:

$c\cdot n^3 \leq \frac{n^3}{6} - 7\cdot n^2$

\noindent Воспользуемся математической индукцией.

\noindent База индукции:

Пусть $n = 1$ и $c = \frac{-41}{6}$, тогда:

$\frac{-41}{6} \leq \frac{-41}{6}$

\noindentИндукционный переход:

Пусть для n неравенсто $c\cdot n^3 \leq \frac{n^3}{6} - 7\cdot n^2$ верно. Распишем его:

\begin{equation}
    c\cdot n^3 \leq \frac{n^3}{6} - 7\cdot n^2
    \Rightarrow
    6c\cdot n \leq n - 42 
    \Rightarrow
    n(6c-1) \leq -42
\end{equation}

Проверим его для n + 1:

\begin{equation*}
    c\cdot (n+1)^3 \leq \frac{(n+1)^3}{6} - 7\cdot (n+1)^2
    \Rightarrow
    6c\cdot (n+1)^3 \leq (n+1)^3 - 42\cdot (n+1)^2
    \Rightarrow
    (n+1)^3(6c - 1) \leq -42\cdot(n+1)^2
    \Rightarrow
\end{equation*}

\begin{equation}
    \Rightarrow
    (n+1)(6c-1) \leq -42
\end{equation}

Из (1) следует то, что $(6c - 1) < 0$. Зная это, (2) неравенсто становиться верным.

Следовательно, 
\begin{equation*}
    \frac{n^3}{6} - 7\cdot n^2 = \Omega(n^3)
\end{equation*}

\section{Задача №5}
    Ответ: 
    \begin{align*}
        1,n^{\frac{1}{logn}}, (\frac{3}{2})^2, loglogn, \sqrt{logn}, log^2n, (\sqrt{2})^{logn}, n, 2^{logn}, nlogn, log(n!),\\
        n^2, 4^{logn}, n^3, log(n)!, (logn)^{logn}, n^{loglogn}, n\cdot 2^n ,e^n, 2^{2^n}, 2^{2^{n + 1}},  n!, (n+1)!
    \end{align*}
\end{document}
