\documentclass{article}
\usepackage[utf8]{inputenc}
\usepackage[russian]{babel}
\usepackage{amsmath}
\usepackage{amssymb}
\usepackage{graphicx}

\title{Lectures DM term2}
\author{Михаил Корнилович}

\begin{document}
    \pagenumbering{gobble}
    \maketitle
    \newpage
    \pagenumbering{arabic}

	\section{Теория вероятности}
	\subsection{Вероятное пр-во}
	$\Omega$ - элементарные исходы (мн-во конечное и счётное) 
	
	\noindent$p$ - дискретное плотность вероятности
	$p: \Omega \to [0, 1]$
	
	$$
	\sum\limits_{\omega \in \Omega}p(w) = 1
	$$
	
	\noindent Событие (случайное) $A \subset \Omega$
	
	\subsection{Примеры}
	\begin{enumerate}
		\item Честная монета $\Omega = \{0, 1\},\ p(0) = p(1) = \frac{1}{2}$.
		\item Нечестная монета $\Omega = \{0, 1\},\ p(1) = p, p(0) = q, p + q = 1$
		\item Честная игральная кость $\Omega = \{1, 2, 3, 4, 5, 6\}, p(\omega) = \frac{1}{6}$
	\end{enumerate}
	
	\subsection{Верояиность события}
	$P(A) = \sum\limits_{\omega \in A}p(\omega) = \mathbb{P}(A) = Pr(A)$ - формула вероятности события.
	
	{\bf Nota bene} 
	
	\noindent Не существет дискретного вероятного пространства с бесконечным числом равновероятных исходов.
	
	\subsection{Независимые события}
	$A$ и $B$ независимы, если $P(A \cap B) = P(A) \cdot P(B)$
	
	\begin{align*}
		P(A \cap B) = P(A) \cdot P(B) \\
		\frac{P(A \cap B)}{P(B)} = \frac{P(A)}{P(\Omega)} \\
		P(A|B) = \frac{P(A \cap B)}{P(B)} \textbf{ - условная вероятность A при условии B}
	\end{align*}
	
	\subsection{Произведение вероятных пространств}
	\begin{align*}
		\Omega_1, p_1 \\
		\Omega_2, p_2 \\
		\Omega = \Omega_1 \times \Omega_2 \\
		p(<\omega_1, \omega_2>) = p_1(\omega_1) \cdot p_2(\omega_2)
	\end{align*}
	
	\noindent $\forall A_1 \subset \Omega_1 \ \forall A_2 \subset \Omega_2$ 
	
	\noindent $A_1 \times \Omega_2, \Omega_1 \times A_2$ - независимы.
	
	{ \bf Доказательство }
	$$
		P(A_1 \times \Omega_2 \cap \Omega_1 \times A_2) = P(A_1 \times A_2) =
		\sum\limits_{a \in A_1, b \in A_2}p(<a, b>) =
		\sum\limits_{a \in A_1}\sum\limits_{b \in B_1}p_1(a)p_2(b) =
	$$
	$$
			\sum\limits_{a \in A_1}p_1(a)(\sum\limits_{b \in A_2}p_2(b)) =
		P_1(A_1)P_2(A_2)
	$$
	
	{ \bf Конец доказательства }

	\subsection{Независимость для более чем 2 событий}
	$A_1A_2..A_n$ - события
	
	\begin{enumerate}
		\item Попарно независимы. $A_i, A_j $ - независимы
		\item Независимы в совокупности. 
		$$
			\forall I \subset \{1, 2, .., n\} P(\bigcap\limits_{i \in I}A_i) = \prod\limits_{i \in I}P(A_i), \quad
			P(A_1 \cap A_2 \cap A_3) = P(A_1)P(A_2)P(A_3)
		$$
	\end{enumerate}
	
	\subsection{Формула полной вероятности}
	$$
		\Omega = A_1 \cup A_2 \cup .. \cup A_n, \quad i \neq j: A_i \cap A_j = \varnothing
	$$
	Полная система событий.
	
	Дано: $P(A_i) \quad P(B|A_i)$
	
	Найти: $P(B)$
	
	$$
		P(B) = \sum\limits_{i = 1}^np(B \cap A_i) =
		\sum\limits_{i = 1}^nP(B|A_i)P(A_i) \textbf{ - формула полной вероятности}		
	$$
	
	\subsection{Формула Байеса}
	
	$$
		P(A_j|B) = \frac{P(A_j \cap B)}{P(B)} =
		\frac{P(B|A_j)P(A_j)}{\sum\limits_{i = 1}^n  p(B|A_i)p(A_i)}
	$$
	
\end{document}